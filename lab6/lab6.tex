%Writeup for ENGR202 Lab 6 
%If editing in vim/vi please insert your own line breaks! This will help keep the file cleaner and easier to edit.

%::Collaboration etiquette::
%In order to better identify changes
%please surround all edits with your
%name as follows:

%BEGIN Zach
%END Zach

%LaTeX will ignore line breaks so to denote
%changes in the middle of a line begin the
%edits on a new line after your name tag
%and then continue with the original on
%the next line following your ending name
%tag.



\documentclass{article}
\usepackage{graphicx}
\graphicspath{ {images/} }

%Change # to correct lab number!
\title{Chapter #6 Lab Writeup}
\author{Zach Thompson, Simon Hannes, Kyle Peterson}
\begin{document}

\maketitle{}

%BEGIN SIMON
\section*{Introduction}
The goal of this Lab is to put together all the information we've amassed 
during the last 9 weeks of class in building a filter network with three
distinct filters catered to specific types of speakers. These filters will
be:
\begin{enumerate}
\item a low pass filter designed for use with a subwoofer
\item a bandpass filter designed for use with the speaker provided in our lab kits
\item a high pass filter designed for use with a tweeter
\end{enumerate}

We then will connect this filter network to appropriate amplification for each 
filter, and out to the speakers. In addition, we will make the system mobile
 and add in a potentiometer to vary the volume, plus an LED to show when the 
network is receiving power.

\section*{Design}
The capacitors available for use were $0.1\mu F$, $1\mu F$ and $10\mu F$. We 
were warned against using resistor values of less than $100\Omega$.

The first consideration we made was determining the frequency range which 
we hoped to encompass by each filter. We used cutoff frequencies of 200Hz
for the low pass, 6kHz for the high pass, and 500Hz, 5kHz for the band-pass.
We used the equation: $Fcutoff = 1/(2\pi*RC)$ to determine capacitor and 
resistor values for each filter. Our initial guesses were shown to be fairly 
functional when modeled in LTSpice, with the exception of the band-pass which 
needed slight modification to achieve an amplitude which matched the amplitude 
created by the other two  filters. To do this, we increased the resistance
in the high-pass section of  the band-pass, and lowered the attached capacitor
by a factor of ten. The values are as follows:
\begin{enumerate}
\item LOW-PASS: $200Hz=795\mu seconds$. We chose a $1\mu$ capacitor and 
$795\Omega$ worth of resistors.
\item HIGH-PASS: $6kHz=27\mu seconds$. We chose a $0.1\mu$ capacitor and 
$270\Omega$ worth of resistors. 
\item BAND-PASS: For the low-end, we found $400Hz=40\mu seconds$. We chose a $0.1\mu$ capacitor and $400\Omega$ worth of resistors. For the high-end, we found
$2kHz=20\mu seconds$. For this we chose a $0.1\mu$ capacitor and $2000\Omega$
worth of resistors.
\end{enumerate}
%END SIMON

\section*{Simulations and Testing}

\section*{Results}

\section*{Conclusion}

%BEGIN SIMON
\section*{Extra Credit}
We added in several extra components to increase the functionality of the 
device including:
\begin{enumerate}
\item An LED to show when the device is powered
\item A fan to deal with heat created by the amps which were found to be in
 excess of 110 degrees.
\item A potentiometer installed before the filters in the signal path to 
control volume.
\end{enumerate}
%END SIMON
\end{document}
